\documentclass{article}
\usepackage[utf8]{inputenc}
\usepackage{mathtools}
\usepackage{amsmath}
\usepackage{amssymb}
\usepackage{witharrows}
\usepackage{cancel}

\title{Operative Research}
\author{Pietro Marcatti}
\date{First Semester 2022/2023}

\begin{document}
    \maketitle
    Teorema: Cnsider a polyhedron
    $P = \left\{ x \in \mathbb{R}^n : Ax\leq b \right\} = \left\{ x: A^=x = b^=, A^<x \leq b^< \right\}$
    It can be shown that $dim(P) = n - rank(A^=)$.
    This implies that:
    \begin{itemize}
        \item a full-dimensional polytope cannot have any implicit equations
        \item a single point cannot have true inequalities and there are n linearly independent equations
    \end{itemize}
    $\alpha^T x \leq \beta$ is a valid inequality for P if satisfied by all $v \in P$. The set $F: H(\alpha, \beta) \intercal P(!=0)$ is called a face of P, defined or induced by the inequality
    $\alpha^T x \leq \beta$
    The most important faces of a polyhedron are those whose dimension is $ = d- 1$ where $d$ is the dimensions of the whole polyhedron, as described before\\
    In the definition of a polyhedron some inequalities may be redundant and could be removed. However the inequalities that result, that create
    facets are never implied by any other inequality and are thus essential. If one is removed the polyhedron changes. \newline
    Let ext(P) be tbhe set of vertices of P. A polyhedron P is pointed if $ext(P) != \emptyset$. Some simple condition for being pointed are:
    \begin{itemize}
        \item P does not contain any line infinite in both directions
        \item P contains the inequalities $x_i \geq 0 for i= 1, \ldots, n$
        \item P is a polytope
    \end{itemize}
    The minkowski-weil theorem states: A set $P \subseteq \mathbb{R}^n $ is a polytope if and only if $P = conv(S)$ for some finite set S of vectors in $\mathbb{R}^n$
    In view of this theorem there are two possible representations of a polytope P:
    \begin{itemize}
        \item external representation: a set of hyperplanes such that P is the intersection of their half-spaces
        \item internal representation: a set of vectros such that P is their convex hull
    \end{itemize}

    \section*{Modelling for Linear Programming}
    Per convenzione in problemi di minimo mettiamo tutti i vincoli $\geq$, viceversa per i massimi mettiamo $\leq$.

    The goal function is a linear function. Each variable $x_i$ has a cost that depends, linearly, on the variable's value. By doubling the value, the cost also doubles and so on. Let $c_i$ be the
    
    For many problems the objective function is non-linear. Start-up costs can be modeled using suitable binary variables.
    The decision is then shifted to how much more than the threshold $\tau$ do i want to produce?
    The meaning of the binary variable z is: "is y>0?". 
    The same reasoning can be applied multiple times to represent economies of scale setting up steps at which each individual unit lowers its cost.
    \subsection*{Linear Constraints}
    A disequality is loose for $(x_i, x_2, x_k) \leftarrow (\gamma_1, \gamma_2, \gamma_k$)

    If a non-negative variable must take only values within a range [L,U], we enforce this condition by two constraints,
    $x \geq L$, $x\leq U$. There might be cases in which we want a variable to be either 0 or within the range [L,U]. The variable must then take values in the set (which is not convex)
    $$X:= \{0\} \cup [L,U]$$
    we can enforce this by adding a new variable y:
    $$x\geq Ly$$
    $$x \leq Uy$$
    Big-M Method: suppose we have a series of OR constraints, these create a union of half-spaces which is not convex.\\
    Let M be "large enough" (a virtual infinity) such that
    $$- M \leq a_i^Tx \leq M$$
    is always true when x satisfies all the remaining constraints of the model. If y is a binary variable, constraints as such:
    $$a^Tx\leq b+My \quad a^Tx\geq b-My$$
    Coming back to the union of half spaces we can introduce k binary variables $y_1, y_2, \ldots y_k$ and k+1 constraints: 
    $$a_i^Tx\leq b_i+ My_i, \forall i=1\ldots k$$
    and 
    $$\sum_{i=1}^{k}{y_i}\leq k-1$$
    This forces at least one of these binary variables to be binding, creating a non-trivial constraints.
    Modelling the absolute value of a variable: let $x \in \mathbb{R}$ To make z be equal to $|x|$
    $$z\geq-x \quad z\geq x$$
    $$z \leq -x+ My\quad z\geq x+M(1-y)$$


    \section*{Linear Programming}
    A linear programming problem is the maximization or minimization of a linear functino over a set of real-valued vectros constrained by a finite number of linear inequalities and/or equations. An instance of a linear programming problem is a linear program.\\
    We can define a matrix A whose rows are the vectors $a_1^T, \ldots , a_m^T$. Assume $|I| = m_1, |E| = m_2, |C|=n_1 and |U|=n_2$. We can partition A:
    $$
    \begin{matrix*}
        A^11, A^12
    \end{matrix*}
    $$

    Canonical Form:
    When E= U = $\emptyset$ we say the LP is in canonical form and can be written in short: $\max\{c^Tx: Ax\leq b, x\geq 0\}$

    Standard Form:
    When I=U=$\emptyset$ we say that the LP is in standard form and can be written in short: $\max\{c^Tx: Ax= b, x\geq 0\}$
    It's always possible to transform a LP in general form into standard from. All inequalities can be turned into equalities by simply adding a slack variable to turn the disequality into an equality.
    Inversely a LP in standard form can be turned into a canonical form LP by splitting the equalities in two symmetric disequalities
\end{document}