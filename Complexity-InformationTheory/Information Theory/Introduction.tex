\section{Introduction}
The information comunicated with a sentence, or any other way, is always dependant on the context. Same goes for the quality for the information. If an event has a low probability of occurring in a given context the information it provides is high, inversely if the probability is high the information is little.\\
We could try to describe information with a simple inverse model:
$$ Information = \frac{1}{p(E)}, \quad \text{E=event}$$
$$p(E)\leadsto 0 \quad Information \leadsto \infty$$
$$p(E)\leadsto 1 \quad Information \leadsto 1$$
Instead if we were to adopt a logaritmic model:
$$Information = \log{\frac{1}{p(E)}}=-\log{p(E)}, \quad \text{E=event}$$
$$p(E)\leadsto 0 \quad Information \leadsto \infty$$
$$p(E)\leadsto 1 \quad Information \leadsto 0$$
Let's consider an alphabet as a group of possible events for which we have a probability distribution:
$$A={a_1,a_2,a_3,\ldots,a_k}$$
$$P={p_1,p_2,p_3,\ldots,p_k}$$
meaning that $p_i$ is the probability to observe the event $a_1$. We can then calculate the average quantity of information as follows:
\begin{equation}
    \sum_{i=1}^{k}{p_i\cdot (-\log{p_i})} = \underset{Shannon Entropy}{\mathbb{H}(P)}
\end{equation}
\subsection{Coin Flip Example}
\textsc{Fair Coin}
$$A=\{H,T\}$$
$$P(H)=\frac{1}{2},\quad P(T)=\frac{1}{2}$$
$$\mathbb{H}(P)=\frac{1}{2}\cdot(-\log{\frac{1}{2}})+\frac{1}{2}\cdot(-\log{\frac{1}{2}})=1$$
Every time we filp the fair coin we expect to gain a bit of information.
\textsc{Unfari Coin}
$$A=\{H,T\}$$
$$P(H)=\frac{9}{10},\quad P(T)=\frac{1}{10}$$
$$\mathbb{H}(P)=\frac{9}{10}\cdot(-\log{\frac{9}{10}})+\frac{1}{10}\cdot(-\log{\frac{1}{10}})\leadsto 0.32$$
\subsection{Properties of Entropy}
\begin{itemize}
    \item $\mathbb{H}(P)\geq 0$
    \item $\mathbb{H}(P) = 0$ if there is an event with probability 1
    \item $\mathbb{H}$ is continuous with respect to $P$
    \item If an event gets split the entropy should be additive
    \item $\mathbb{H}(p_1,p_2,\ldots,p_k)\leq \log{k}= \mathbb{H}(\frac{1}{k},\frac{1}{k},\ldots,\frac{1}{k})$
\end{itemize}
To prove the last property we can first show that for a equi-distribution we get $\mathbb{H}=\log{k}$. To prove that this is also the maximum we use Jensen disequality:
$$\forall f, f''(x)<0 \quad in\;[a,b], (a,b)\in \mathbb{R}^2$$
$$f\Big{(}\sum_{i=1}^{k}{\lambda_i\cdot x_i}\Big{)}\geq\sum_{i=1}^{k}{\lambda_i\cdot f(x_i)}, \quad \sum{\lambda_i} = 1$$

